%----------------------------------------------------------------------------
\chapter{Kérdőíves felmérések módszertana}

A kérdőívnek alapvetően kétféle tétele van: kérdések és állítások. A kérdő és a kijelentő forma egyaránt hasznos lehet, kombinálásukkal elkerülhető a monotómia.
\begin{itemize}
\item \emph{Kérdések:} a kérdések sorrendje hatással lehet a későbbi válaszokra, ezért az általánostól érdemes indulni a konkrét felé és témakörönként csoportosítani kell őket. Önkitöltős tesztnél fontos az is, hogy előbb érdekes kérdések jöjjenek, különben a kérdezett elunja magát és a kérdőív nagy részét nem tölti ki. A kérdések sorrendjének randomizálása megnehezítheti a kitöltést, ezért előre fel kell mérni, hogy melyik kérdés vezetheti meg a kérdezettet egy később felteendő kérdésnél. Ha van ilyen, azt hátrébb kell sorolni
	\begin{itemize}
	\item A kérdések legyenek egyértelműek, világosak, csak egy dologra kérdezzenek rá, relevánsak legyenek és ne sugalmazzák a választ
	\item A kérdezettnek kompetensnek kell(ene) lennie és hajlandónak válaszolni
	\item \emph{Nyitott kérdés:} a válaszoló a saját szavaival fogalmazza meg a választ. A nyitott kérdések előnye, hogy nem köti meg a kérdezett fantáziáját, viszont nehezen kódolható, ráadásul teret ad a kutató szabad értelmezésének, amit torzítást eredményezhet. Előfordul, hog a válasz irreleváns
	\item Zárt kérdések esetében a válaszolónak egy listából kell kiválasztania a lehetséges válaszokat. Fontos, hogy a válaszok teljes eseményrendszert alkossanak. A zárt kérdéseket könnyen és egyértelműen lehet kódolni számítógépes feldolgozáshoz, de figyelni kell a hiányzó válaszok feldolgozásánál. Zárt kérdéseknél előfordulhat, hogy a kérdezett több választ is meg tudna jelölni, ráadásul a gyűtő válasz ("egyéb") nagyon tág lehet
	\item \emph{Feltételes kérdések:} elágazási pontok a kapott választól föggően
	\item \emph{Mátrix kérdések:} ha egy csoportban teszünk fel zárt kérdéseket Likert-skálás válaszokkal, mátrix struktúrát alkalmazhatunk (táblázatos forma: sorok a kérdések, oszlopok a lehetséges válaszok)
	\end{itemize}
\item \emph{Állítások:} akkor alkalmazzuk, ha a kutató azt akarja megtudni, hogy milyen mértékben oszt a kérdezett bizonyos attitűdöt vagy nézetet. Az attitűdöt egy tömör kijelentésben összefoglaljuk, és megkérdezzük, mennyiben ért ezzel egyet a kérdezett.
\end{itemize}

A lehetséges válaszokat Likert formalizálta, megalkotva a \emph{Likert-skálát}. Likert-skála esetén a válaszolónak egy állítással való egyetértés mértékét, vagy egy vélemény helyeslését kell kifejeznie.  A kérdőívszerkesztőnek csak az állítást kell meghatároznia, maga a skála mindig ugyanaz, pl. Egyetért - Közömbös - Nem ért egyet. A skála lehetnek
\begin{itemize}
\item Verbális/nem verbális
\item Egyirányú: pl. 5 = teljes mértékben egyetért, ..., 1 = egyáltalán nem ért egyet
\item Középre rendezett: pl. 5 = teljesen elégedett, ..., 3 = közömbös, ... 1= teljesen elégedett
\item Szemantikus differenciál: az intenzitást és a tartalmat egyszerre vizsgálja a megkérdezett gondolkodásmódjában egy hétfokozatú skálán, pl. korszerű 1,2,...,6,7 régimódi. Így egyéni és csoportátlagok, szóródások számíthatók
\end{itemize}
A skálaértékek között egyenletes távolságoknak kell lenniük, biztosítani kell a megfelelő szórást és szemantikus differenciál esetén a verbális végpontoknak tartalmilag szembenállóknak kell lenniük

\emph{Primer adatok:} a mintavételből nyert adatok.

\emph{Szekunder adatok:} meglévő adatázisokból, releváns forrásokból és a szakirodalomból nyert adatok.

\section{Adatgyűjtési technikák}

\emph{Interjú módszer:} az interjú módszer korlátozza a kutató előítéletéből fakadó korlátokat, interaktív és lehetőséget ad a változtatásra. Személyesen vagy telefonon történik jegyzeteléssel vagy hangrögzítéssel. Lehet:
\begin{itemize}
\item Strukturálatlan: szabad beszélgetés (nehéz a kódolása)
\item Dinamikus, non-direktív: ilyen csak irányító kérdések vannak, a kérdezőnek nem szabad közbekérdezni
\item Strukturált: irányított beszélgetés, ami kódolható és statisztikai feldolgozásra alkalmas
\end{itemize}

\emph{Kérdőíves adatgyűjtés:} sikere a kérdések megfogalmazásán áll vagy bukik. Kutatási kérdések megválaszolását szolgálja, a válaszok alapján kell tudni dönteni a hipotézisről. Lépései:
\begin{enumerate}
\item Kérdőív-szerkesztés: ne legyen túl rövid, elrendezésnél fekvő és álló is elfogadható, legyen rajta verziószám és a papírnak csak az egyik oldalára nyomtassunk. A kérdéseket és válaszokat számozzuk, az utasításokat CSUPA NAGY BETŰVEL ÍRJUK. Ha 5 vagy kevesebb választási lehetőség van egy zárt kérdésnél, akkor azokat két függőleges oszlopba kell rendezni, kerüljük a tagadó kérdéseket, negatív megfogalmazást
\item Kérdőív tesztelése
\item Mintavétel (lásd \ref{ch:mintavetel}. tétel)
\item Adatgyűjtés: lehet személyes megkérdezés, telefonos felvétel vagy önkitöltős kérdőív, lehetőség van másodelemzésekre. Pl. önkitöltő kérdőív fókuszcsoportban (célcsoport közös beszélgetésen vesz rész, ált. 8 főből áll), vezetőknél kitöltött kérdezőbiztosi, kombinált telefon/posta, stb.
\item Kiértékelés
\end{enumerate}
